\documentclass{article}

\usepackage[brazil]{babel}
\usepackage[T1]{fontenc}
\usepackage[a4paper, margin=1.5cm]{geometry}
\usepackage[colorlinks, urlcolor=blue]{hyperref}
\usepackage[utf8]{inputenc}
\usepackage{enumitem, mathtools}

\pagestyle{empty}

\newenvironment{arabenum}{
    \begin{enumerate}[label=\textbf{\arabic*})]
}{
    \end{enumerate}
}

\newenvironment{alphenum}{
    \begin{enumerate}[label=(\alph*)]
}{
    \end{enumerate}
}

\begin{document}

{\bf \noindent INE5409 - Cálculo Numérico para Computação (2016/1) \\
Gustavo Zambonin \\
Trabalho 1 - Métodos diretos e iterativos
para sistemas esparsos de equações lineares \\
}

\noindent \textbf{Nota}: todos os algoritmos utilizados podem ser encontrados
também \href{https://github.com/zambonin/ufsc-ine5409}{neste repositório}.

\begin{arabenum}

\item \begin{alphenum}

\item A solução para o sistema de equações lineares fornecido, a partir de um
método direto, é apresentada abaixo. O método escolhido, em virtude das
características do sistema linear, foi a eliminação Gaussiana simples.
Por conta da diferença de comportamento do operador de divisão
(\texttt{/}) entre versões da linguagem escolhida (Python), o ambiente
virtual de programação gerará resultados diferentes, mas que também
estarão corretos.

\begin{verbatim}
[-2.8561706131735374, 4.356170613173537, -4.113756834522826, 4.038439287217356,
-3.373920443943568, 3.1855001854512595, -2.4906755620531884, 2.2941348942505835,
-1.5971344137356092, 1.4000107267756854, -0.7028540267483352, 0.5056884808962286,
0.19147943518747773, -0.3886479863728271, 1.0858167077331484, -1.282985474691716,
1.9801542538682972, -2.1773230363186853, 2.8744918196462863, -3.071660603208936,
3.7688293868345686, -3.9659981704770764, 4.663166954124104, -4.860335737772341,
5.557504521420897, -5.754673305069517, 6.4518420887180685, -6.649010872366281,
7.346179656013208, -7.543348439655333, 8.24051722327953, -8.43768600683683,
9.134854790144445, -9.332023572520237, 10.02919235141841, -10.226361117337895,
10.923529834820279, -11.120698371532933, 11.817866233603752, -12.015031577876963,
12.712187525601552, -12.909308404929252, 13.606298407217995, -13.802799969747818,
14.49747865028091, -14.685354242585968, 15.347840363975656, -15.415571689931879,
15.629673305069653, -14.024010872367091, 7.992921959121624, -5.098584391824248,
4.946660603177585, -3.6276405831856757, 3.897821859162089, -2.6919045503570223,
2.9923916064577214, -1.7945947069629535, 2.097257620180558, -0.9000437398431117,
1.2028628725730912, -0.005690851127828677, 0.3085211999141652, 0.888647816197855,
-0.5858166621349022, 1.7829854624737023, -1.4801542505944907, 2.677323035441472,
-2.3744918194112374, 3.571660603145953, -3.2688293868176923, 4.465998170472556,
-4.163166954122894, 5.360335737772022, -5.057504521420833, 6.254673305069585,
-5.951842088718408, 7.149010872367566, -6.846179656018012, 8.04334843967326,
-7.740517223346434, 8.937686007086507, -8.63485479107627, 9.832023575997864,
-9.52919236439709, 10.726361165774996, -10.42353001559001, 11.620699046174765,
-11.31786875140136, 12.515040974425588, -12.212222593998442, 13.409439281968202,
-13.106786846976911, 14.304622851744554, -14.00428173850894, 15.210743713501344,
-14.94259515940912, 16.26920140075033, -16.449437352909992, 19.449437352909996]
\end{verbatim}

\item Considera-se $n = 100$ o número de equações do sistema,
e $c$ uma unidade de computação.

\begin{itemize}

\item Para somas, tem-se
\begin{equation*}
\sum_{i=1}^n \sum_{j=i+1}^n c = \frac{c}{2} (n-1)n = \boldsymbol{4950}c
\end{equation*}

\item Para subtrações, tem-se
\begin{equation*}
c \cdot n + \sum_{i=1}^n \sum_{j=i+1}^n c
+ \sum_{i=1}^n \sum_{j=i+1}^n \sum_{k=i+1}^n c
= c \cdot n \cdot \Big( \frac{1}{6}(2n^2 - 3n + 1) + \frac{1}{2}(n-1) + 1 \Big)
= \boldsymbol{333400}c
\end{equation*}

\item Para multiplicações, tem-se
\begin{equation*}
2 \cdot \Big( \sum_{i=1}^n \sum_{j=i+1}^n c \Big)
+ \sum_{i=1}^n \sum_{j=i+1}^n \sum_{k=i+1}^n c
= c \cdot n \cdot \Big( (n-1) + \frac{1}{6}(2n^2 - 3n + 1) \Big)
= \boldsymbol{338250}c
\end{equation*}

\item Para divisões, tem-se
\begin{equation*}
c \cdot n + \sum_{i=1}^n \sum_{j=i+1}^n c
= c \cdot n \cdot \Big( \frac{1}{2}(n-1) + 1 \Big) = \boldsymbol{5050}c
\end{equation*}

\end{itemize}

Então, o número final de operações é de
\begin{equation*}
c \cdot n \cdot \Big( \frac{5}{2}(n-1) + \frac{1}{3}(2n^2 - 3n + 1) + 2 \Big)
= \boldsymbol{681650}c.
\end{equation*}

\end{alphenum}

\item \begin{alphenum}

\item A análise de convergência de um sistema deste tipo é realizada pela
verificação da dominância diagonal da matriz, ou seja, uma matriz $A$ é
diagonalmente dominante se
\begin{equation*}
\vert a_{ii} \vert \geq \sum_{j \neq i} \vert a_{ij} \vert \text{ para todo } i
\end{equation*}
onde $a_{ij}$ denota o elemento da linha $i$ e coluna $j$. O método
\texttt{check\_diagonal\_dominance} pode ser executado para verificar isto.
Neste caso, como seu resultado é \verb!False!, nada se pode afirmar sobre a
convergência desse sistema linear, e assim, fatores de relaxação devem ser
testados.

\item Em virtude da escolha do épsilon de máquina $(\epsilon)$ como tolerância
padrão para a parada de iterações, é possível efetuar um grande número de
iterações e visualizar um fator de relaxação $(\omega)$ que gere melhores
resultados. Primeiramente, foi-se testado o intervalo total de fatores válidos
$(0 < \omega < 2)$ com um número de iterações baixo ($100$), e percebeu-se
um grande número de decimais exatos após à virgula perto do número
$\boldsymbol{1.80}$. Diminuindo o intervalo e aumentando o número de iterações
para $1000$, concluiu-se que $\omega = \boldsymbol{1.878}$ é um bom valor,
com precisão de 13 casas após à vírgula com tolerância $= \epsilon$ na primeira
equação do sistema linear.

\item O resultado abaixo é calculado com $\omega = 1.878$ e
$\epsilon = 1 \cdot 10^{-4}$.

\begin{verbatim}
Maximum tolerance exceeded at iteration 310.
[-2.853251748355784, 4.353297944089141, -4.110850693888152, 4.035486043454446,
-3.370924423948766, 3.1824737836481996, -2.487629628286154, 2.2910800962042974,
-1.59408102272795, 1.396967795131335, -0.6998334460019331, 0.5026990446137996,
0.19442566710570072, -0.391542346947283, 1.0886487463096102, -1.2857456065699253,
1.9828343068555485, -2.1799122759627085, 2.8769841726516505, -3.0740453980103237,
3.77110135318818, -3.968148581980649, 4.665189654771799, -4.8622254325847925,
5.559253450678749, -5.756279610488767, 6.4532974437566075, -6.650315037453679,
7.347326222216086, -7.544336174290529, 8.24134407333501, -8.438348022490775,
9.13535484341079, -9.332355875425623, 10.02936269622345, -10.226364968659265,
10.923371772494486, -11.120379076520408, 11.817386534651924, -12.01439914786595,
12.711398332665818, -12.908374202495835, 13.605216179853931, -13.801582555893958,
14.496127099177926, -14.683870565700358, 15.346233828151236, -15.41384005329251,
15.62781629739698, -14.021932566540576, 7.9903334021509735, -5.09581371310809,
4.943804189390155, -3.6247290257569396, 3.894867718130789, -2.6889195307384313,
2.9893859242958825, -1.7915815132133723, 2.0942468817601316, -0.8970465553452366,
1.1998890193728338, -0.0027496670578067398, 0.30562382141222044, 0.8914933214165963,
-0.5885987280483297, 1.7856967653589122, -1.482784389711428, 2.6798641016476066,
-2.3769358033407126, 3.573998277219781, -3.2710554843163733, 4.468102832143591,
-4.165146596120273, 5.362181729270814, -5.0592130021293285, 6.256238927216705,
-5.953259245360462, 7.150278330280135, -6.847290297093205, 8.044303891127917,
-7.741311130609901, 8.938320037946559, -8.635326337821088, 9.832331586125056,
-9.52933982463239, 10.72634404334204, -10.423355750269597, 11.620362793388969,
-11.317378410344569, 12.51439607219345, -12.211427821404676, 13.408500734741132,
-13.105702253720045, 14.303405672048342, -14.002930085527035, 15.209273755129649,
-14.941003835449457, 16.26750152361424, -16.447664759134533, 19.447682337305743]
\end{verbatim}

\item Considera-se $n = 100$ o número de equações do sistema, $s = 310$ o
número de iterações e $c$ uma unidade de computação.

\begin{itemize}

\item Para somas, tem-se
\begin{equation*}
\sum_{i=1}^s \sum_{i}^n c = c \cdot n \cdot s = \boldsymbol{31000}c
\end{equation*}

\item Para subtrações, tem-se
\begin{equation*}
2 \cdot \sum_{i=1}^s \sum_{i}^n c = 2 \cdot c \cdot n \cdot s
= \boldsymbol{62000}c
\end{equation*}

\item Para multiplicações, tem-se
\begin{equation*}
\sum_{i=1}^s \Big(\sum_{i}^n\Big)^2 c = c \cdot n^2 \cdot s
= \boldsymbol{3100000}c
\end{equation*}

\item Para divisões, tem-se
\begin{equation*}
\sum_{i=1}^s \sum_{i}^n c = c \cdot n \cdot s = \boldsymbol{31000}c
\end{equation*}

\end{itemize}

Então, o número final de operações é de
\begin{equation*}
c \cdot n \cdot s \cdot (n + 4) = \boldsymbol{3224000}c.
\end{equation*}

\item O maior erro de truncamento relativo foi calculado a partir da diferença
entre as soluções obtidas com $s = 1000$ e $s = 2000$:
$\boldsymbol{1.2350565015140091 \cdot 10^{-11}}$.

\end{alphenum}

\end{arabenum}

\end{document}
