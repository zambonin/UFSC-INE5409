\documentclass{article}

\usepackage[brazil]{babel}
\usepackage[T1]{fontenc}
\usepackage[a4paper, margin=1.5cm]{geometry}
\usepackage[colorlinks, urlcolor=blue]{hyperref}
\usepackage[utf8]{inputenc}
\usepackage{enumitem, mathtools}

\newenvironment{arabenum}{
    \begin{enumerate}[label=\textbf{\arabic*})]
}{
    \end{enumerate}
}

\newenvironment{alphenum}{
    \begin{enumerate}[label=(\textit{\alph*})]
}{
    \end{enumerate}
}

\title{\textbf{Métodos diretos e iterativos
para sistemas esparsos de equações lineares}}
\author{Gustavo Zambonin\thanks{\texttt{gustavo.zambonin@grad.ufsc.br} ---
todos os algoritmos utilizados podem ser encontrados
também \href{https://github.com/zambonin/ufsc-ine5409}{neste repositório}.} \\
\small {Cálculo Numérico para Computação (UFSC -- INE5409)} \vspace{-5mm}}
\date{}

\begin{document}

\maketitle

\begin{arabenum}

\item \begin{alphenum}

\item A solução para o sistema de equações lineares fornecido, a partir de um
método direto, é apresentada abaixo. O método escolhido, em virtude das
características do sistema linear, é a eliminação Gaussiana simples.

\begin{verbatim}
[-2.8561706131735374, 4.356170613173537, -4.113756834522826, 4.038439287217356,
-3.373920443943568, 3.1855001854512595, -2.4906755620531884, 2.2941348942505835,
-1.5971344137356092, 1.4000107267756854, -0.7028540267483352, 0.5056884808962286,
0.19147943518747773, -0.3886479863728271, 1.0858167077331484, -1.282985474691716,
1.9801542538682972, -2.1773230363186853, 2.8744918196462863, -3.071660603208936,
3.7688293868345686, -3.9659981704770764, 4.663166954124104, -4.860335737772341,
5.557504521420897, -5.754673305069517, 6.4518420887180685, -6.649010872366281,
7.346179656013208, -7.543348439655333, 8.24051722327953, -8.43768600683683,
9.134854790144445, -9.332023572520237, 10.02919235141841, -10.226361117337895,
10.923529834820279, -11.120698371532933, 11.817866233603752, -12.015031577876963,
12.712187525601552, -12.909308404929252, 13.606298407217995, -13.802799969747818,
14.49747865028091, -14.685354242585968, 15.347840363975656, -15.415571689931879,
15.629673305069653, -14.024010872367091, 7.992921959121624, -5.098584391824248,
4.946660603177585, -3.6276405831856757, 3.897821859162089, -2.6919045503570223,
2.9923916064577214, -1.7945947069629535, 2.097257620180558, -0.9000437398431117,
1.2028628725730912, -0.005690851127828677, 0.3085211999141652, 0.888647816197855,
-0.5858166621349022, 1.7829854624737023, -1.4801542505944907, 2.677323035441472,
-2.3744918194112374, 3.571660603145953, -3.2688293868176923, 4.465998170472556,
-4.163166954122894, 5.360335737772022, -5.057504521420833, 6.254673305069585,
-5.951842088718408, 7.149010872367566, -6.846179656018012, 8.04334843967326,
-7.740517223346434, 8.937686007086507, -8.63485479107627, 9.832023575997864,
-9.52919236439709, 10.726361165774996, -10.42353001559001, 11.620699046174765,
-11.31786875140136, 12.515040974425588, -12.212222593998442, 13.409439281968202,
-13.106786846976911, 14.304622851744554, -14.00428173850894, 15.210743713501344,
-14.94259515940912, 16.26920140075033, -16.449437352909992, 19.449437352909996]
\end{verbatim}

\item Considera-se $n$ o número de equações do sistema, e $c$ uma unidade de
computação.

\begin{itemize}

\item Para somas, tem-se
\begin{equation*}
\sum_{i=1}^n \sum_{j=i+1}^n c = c \cdot n \cdot \Big( \frac{1}{2}(n-1) \Big)
\end{equation*}

\item Para subtrações, tem-se
\begin{equation*}
c \cdot n + \sum_{i=1}^n \sum_{j=i+1}^n c
+ \sum_{i=1}^n \sum_{j=i+1}^n \sum_{k=i+1}^n c
= c \cdot n \cdot \Big( \frac{1}{6}(2n^2 - 3n + 1) + \frac{1}{2}(n-1) + 1 \Big)
\end{equation*}

Note que é possível remover a primeira iteração da retrosubstituição e calcular
a solução manualmente, já que o somatório de coeficientes à direita será igual
a $0$. Isto elimina uma subtração desnecessária.

\item Para multiplicações, tem-se
\begin{equation*}
2 \cdot \Big( \sum_{i=1}^n \sum_{j=i+1}^n c \Big)
+ \sum_{i=1}^n \sum_{j=i+1}^n \sum_{k=i+1}^n c
= c \cdot n \cdot \Big( (n-1) + \frac{1}{6}(2n^2 - 3n + 1) \Big)
\end{equation*}

\item Para divisões, tem-se
\begin{equation*}
c \cdot n + \sum_{i=1}^n \sum_{j=i+1}^n c
= c \cdot n \cdot \Big( \frac{1}{2}(n-1) + 1 \Big)
\end{equation*}

\end{itemize}

Então, o número final de operações é de
\begin{equation*}
c \cdot n \cdot \Big( \frac{5}{2}(n-1) + \frac{1}{3}(2n^2 - 3n + 1) + 2 \Big).
\end{equation*}

\end{alphenum}

\item \begin{alphenum}

\item A análise de convergência de um sistema deste tipo é realizada pela
verificação da dominância diagonal da matriz, ou seja, uma matriz $A$ é
diagonalmente dominante se
\begin{equation*}
\vert a_{ii} \vert > \sum_{\substack{j=1 \\ j \neq i}}^n \vert a_{ij} \vert
\; \forall \; i = 1, \dots, n
\end{equation*}
onde $a_{ij}$ denota o elemento da linha $i$ e coluna $j$. O método
\texttt{check\_diagonal\_dominance} pode ser executado para verificar isto.
Neste caso, como seu resultado é \verb!False!, nada se pode afirmar sobre a
convergência desse sistema linear, e assim, fatores de relaxação devem ser
testados.

\item Em virtude da escolha do épsilon de máquina de precisão dupla $(\epsilon
\approx 2.22 \cdot 10^{-16})$ como tolerância padrão para a parada de iterações
no método SOR (\emph{sucessive over-relaxation}), é possível efetuar um grande
número de iterações e visualizar um fator de relaxação $(\omega)$ que gere
melhores resultados. Primeiramente, foi-se testado o intervalo total de fatores
válidos $(0 < \omega < 2)$ com um número de iterações baixo ($100$), e percebeu-
se um grande número de decimais exatos após à virgula perto do número
$\boldsymbol{1.80}$. Diminuindo o intervalo e aumentando o número de iterações
para $1000$, concluiu-se que $\omega = \boldsymbol{1.879}$ é um bom valor, com
precisão de 13 casas após à vírgula com tolerância máxima e um grande número de
iterações $s$ (na casa dos milhares) na primeira equação do sistema linear.

\item O resultado abaixo é calculado com $\omega = 1.879$,
$\epsilon = 1 \cdot 10^{-4}$ e $s = 500$.

\begin{verbatim}
[-2.856174712480373, 4.356175656586068, -4.113767927141613, 4.0384578415033525,
-3.373946832412452, 3.1855342893658825, -2.490717332327044, 2.2941842583431327,
-1.5971912946745552, 1.4000750775189341, -0.7029256009703317, 0.5057671741869292,
0.19139395829268932, -0.3885558984779885, 1.0857183334162417, -1.282881121935825,
1.9800441982996733, -2.1772077402425247, 2.8743715188488084, -3.071535846551343,
3.7687004453832307, -3.96586556908092, 4.663031099116311, -4.860197038404439,
5.557363563375291, -5.7545303749205035, 6.451697891208179, -6.648865672014814,
7.346034127013814, -7.543202974714648, 8.240372313905338, -8.437542274642249,
9.134712498148414, -9.33188351828323, 10.029054676235443, -10.226226607358544,
10.923398745795158, -11.120571231680495, 11.817743560650708, -12.01491352184236,
12.712074931391198, -12.90920119764235, 13.606197403951445, -13.802705040272082,
14.497390396055382, -14.685273175562498, 15.34776658154788, -15.415505927929194,
15.629616127017108, -14.023968294604092, 7.992910746001128, -5.0985869651569935,
4.946672187404567, -3.627660085780336, 3.8978491802368667, -2.6919396064705703,
2.9924343670396847, -1.7946449397852384, 2.0973152507581747, -0.900108589648829,
1.2029348102632038, -0.005769726850836549, 0.30860672543611967, 0.8885557928525473,
-0.5857185369961594, 1.7828814221301357, -1.4800447090327213, 2.6772082953537426,
-2.3743722409018617, 3.5715366519587306, -3.2687013455809155, 4.4658666306216075,
-4.163032179716995, 5.360198336515029, -5.057364852259886, 6.254531866503875,
-5.951699421351235, 7.14886730835849, -6.8460359041518055, 8.043204730700085,
-7.740374297511884, 8.9375441536703, -8.634714597883152, 9.831885550924712,
-9.529056820661879, 10.726228844410288, -10.423401009035116, 11.620574196964796,
-11.3177481556616, 12.514925259208868, -12.212112072922006, 13.40933422214768,
-13.106687999503325, 14.304530053474275, -14.0041957421782, 15.210664199370601,
-14.942522976379943, 16.269136433854502, -16.449378206596013, 19.44937955528533]
\end{verbatim}

\item Considera-se $n$ o número de equações do sistema, $s$ o número de
iterações, e $c$ uma unidade de computação.

\begin{itemize}

\item Para somas, tem-se
\begin{equation*}
2 \cdot c \cdot s + \sum_{i=1}^s \Big( 6 \cdot \sum_{j=2}^{n/2} \Big) c
= c \cdot s \cdot (3n - 4)
\end{equation*}

\item Para subtrações, tem-se
\begin{equation*}
2 \cdot \sum_{i=1}^s \sum_{i}^n c = 2 \cdot c \cdot n \cdot s
\end{equation*}

\item Para multiplicações, tem-se
\begin{equation*}
\sum_{i=1}^s \sum_{i}^n c = c \cdot n \cdot s
\end{equation*}

\item Para divisões, tem-se
\begin{equation*}
\sum_{i=1}^s \sum_{i}^n c = c \cdot n \cdot s
\end{equation*}

\end{itemize}

Então, o número final de operações é de
\begin{equation*}
c \cdot s \cdot (7n - 4).
\end{equation*}

\item O maior erro de truncamento relativo foi calculado a partir da diferença
($\big\vert \frac{VA - VE}{VE} \big\vert$ onde $VA$ é o resultado do método SOR,
e $VE$ é o resultado da eliminação Gaussiana) entre as soluções obtidas acima:
$0.0138600924952 \approx \boldsymbol{1.386 \cdot 10^{-2}}$.

\end{alphenum}

\end{arabenum}

\end{document}
