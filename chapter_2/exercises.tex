\documentclass[12pt]{article}

\usepackage[brazil]{babel}
\usepackage[T1]{fontenc}
\usepackage[a4paper, margin=1.5cm]{geometry}
\usepackage[colorlinks, urlcolor=blue]{hyperref}
\usepackage[utf8]{inputenc}
\usepackage{amsmath, enumitem}

\newenvironment{smallitem}{
    \vspace{-2mm}
    \begin{enumerate}
    \setlength{\parskip}{0pt}
    \setlength{\itemsep}{2pt}
}{
    \vspace{-2mm}
    \end{enumerate}
}

% http://tex.stackexchange.com/a/58145
\newcommand\Item[1][]{
    #1 \item
    \abovedisplayskip=0pt
    ~\vspace*{-\baselineskip}
}

\begin{document}

{\bf \noindent Universidade Federal de Santa Catarina - UFSC \\
Centro Tecnológico - CTC \\
Departamento de Informática e Estatística - INE \\

\noindent Lista de Exercícios 1 - INE5202/INE5232/INE5409 \\
Prof. Sérgio Peters (\texttt{sergio.peters@ufsc.br})}

\begin{enumerate}[label=\textbf{\arabic*})]

\item Classifique as possíveis soluções dos seguintes sistemas de equações
lineares, através do método de Gauss com pivotamento total:

\begin{smallitem}

\Item \begin{align*}
\begin{cases}
3 x_1 + 1.5 x_2 + 4.75 x_3 = 8 \\
4 x_1 + 2 x_2 + 3 x_3 = 7 \\
2 x_1 + 5 x_2 + 3 x_3 = -12
\end{cases}
\end{align*}

\Item \begin{align*}
\begin{cases}
x_1 + 0.5 x_2 - 1.75 x_3 = -1 \\
3 x_1 + 1.5 x_2 + 4.75 x_3 = 8 \\
4 x_1 + 2 x_2 + 3 x_3 = 7
\end{cases}
\end{align*}

\Item \begin{align*}
\begin{cases}
x_1 + 0.5 x_2 - 1.75 x_3 = 0 \\
3 x_1 + 1.5 x_2 + 4.75 x_3 = 8 \\
4 x_1 + 2 x_2 + 3 x_3 = 7
\end{cases}
\end{align*}

\end{smallitem}

\item Resolva o sistema de equações lineares abaixo pelo método de Gauss
utilizando pivotação parcial e pivotação total.
\begin{align*}
\begin{cases}
4 x_1 + x_2 + 2 x_3 = -1 \\
x_1 - 2 x_2 + x_3 = 10 \\
x_1 + 0.1 x_2 - x_3 = -3
\end{cases}
\end{align*}

\item Calcule o determinante da matriz de coeficientes utilizando o processo de
eliminação adotado pelo método de Gauss.

\item Monte um algoritmo que calcule o determinante da matriz de coeficientes
escalonada, utilizando o processo de eliminação adotado pelo método de Gauss.

\item (opcional) Elabore um algoritmo para resolver um sistema de equações
lineares pelo método de Gauss-Jordan utilizando pivotação parcial. (Sugestão:
estenda as eliminações do algoritmo de Gauss para as linhas acima e abaixo da
diagonal principal).

\item Dado o sistema de equações lineares:
\begin{align*}
\begin{cases}
3 x_1 + 15 x_2 + 4.75 x_3 = 8 \\
4.01 x_1 + x_2 + 3 x_3 = 7 \\
x_1 + 0.5 x_2 - 0.05 x_3 = -1
\end{cases}
\end{align*}

\begin{smallitem}

\item Resolva-o pelo método de eliminação de Gauss sem pivotamento;

\item Resolva-o pelo método de eliminação de Gauss com pivotamento parcial;

\item Resolva-o pelo método de eliminação de Gauss com pivotamento total;

\item (opcional) Resolva-o pelo método de eliminação de Gauss-Jordan com
pivotamento parcial;

\item (opcional) Resolva-o pelo método de Inversão de matriz com pivotamento
parcial.

\end{smallitem}

\item (opcional) Monte um algoritmo para determinar a matriz inversa de A,
recorrendo ao método de eliminação de Gauss-Jordan com pivotamento parcial.

\item Resolva o seguinte sistema de equações lineares pelo método de Crout:
\begin{align*}
\begin{cases}
4 x_1 - x_2 + x_3 = 0 \\
- x_1 + 4.25 x_2 + 2.75 x_3 = 1 \\
x_1 + 2.75 x_2 + 3.5 x_3 = -1
\end{cases}
\end{align*}

\item (opcional) Resolva o sistema de equações lineares acima pelo método de
Cholesky.

\item Para resolver um sistema de equações lineares da ordem de $n = 10$ (10
equações com 10 incógnitas) em computador, utilizando o método de Eliminação de
Gauss sem pivotamento, a solução foi encontrada em 0.1 segundos de CPU, por
exemplo. Estime o tempo que este mesmo computador levaria para resolver um
sistema da ordem de $n = 1000$ equações, utilizando o mesmo método.

\item Avalie o condicionamento do sistema abaixo pelos dois critérios
estabelecidos:
\begin{align*}
\begin{cases}
x_1 + 0.5 x_2 - 0.05 x_3 = -1 \\
3 x_1 + 1.5 x_2 + 4.75 x_3 = 8 \\
4.01 x_1 + 2 x_2 + 3 x_3 = 7
\end{cases}
\end{align*}

\item Monte um algoritmo genérico \verb!x = fsubstituicoes(n, LU, b)!, que
determine e retorne a solução $x$ do sistema linear $A \cdot x = b$, a partir da
matriz $LU$ decomposta de $A$, onde $LU$ contém $L$ e $U$ concatenadas na mesma
matriz, e do vetor $b$, fazendo as duas substituições $L \cdot c = b$ e $U \cdot
x = c$.
\begin{align*}
LU = \begin{bmatrix}
L_{11} & U_{12} & U_{13} \\
L_{21} & L_{22} & U_{23} \\
L_{31} & L_{32} & L_{33}
\end{bmatrix}
\end{align*}

Determine o número de operações \verb!x = fsubstituicoes(n, LU, b)!, em função
de $n$.

\textbf{Dada a função}: \verb![LU b] = fLU(n, A, b)! - que retorna a matriz LU
decomposta (LU de Crout), com as matrizes $L$ e $U$ armazenadas juntas, com
pivotação parcial interna, através da função
\verb![A b] = fpivotacao(k, n, A, b)! - que troca linhas da matriz $A$ e do
vetor $b$ e retorna do lado esquerdo da função \verb!fpivotacao.m!, uma nova
matriz $A$ com a linha $k$ contendo o maior coeficiente em módulo na coluna $k$.

\item Dado o sistema linear abaixo para $n = 10000$ equações:
\begin{align*}
\begin{cases}
x_i + x_{i + 1} = 150 &
i = 1 \\
x_{i - 1} + 9x_i + x_{i + 1} + x_{i + 100} = 100 &
i = 2, \dots, \frac{n}{2} \\
x_{i - 100} + x_{i - 1} + 9x_i + x_{i + 1} = 200 &
i = \frac{n}{2} + 1, \dots, n - 1 \\
x_{i - 1} + x_i = 300 &
i = n
\end{cases}
\end{align*}

\begin{smallitem}

\item Se o sistema for resolvido por métodos iterativos, como Jacobi ou
Gauss--Seidel, a sua convergência para a solução é garantida? Justifique;

\item É recomendo testar a utilização de fatores de sub-relaxação? Justifique;

\item Monte um algoritmo otimizado, que calcule e imprima a solução $x$ do sistema
linear acima com 10 dígitos significativos exatos, para $n = 10000$ incógnitas, pelo
método que efetue o menor número de operações aritméticas em ponto flutuante.
Justifique a escolha do método adotado.

\end{smallitem}

\item \begin{smallitem}

\item Que cuidados devem ser tomados ao se resolver sistemas mal-condicionados
por métodos diretos. Seria indicado resolver sistemas mal-condicionados por
métodos iterativos? Justifique;

\item Monte um algoritmo otimizado \verb!A = fescalona(n, A)!, que determine e
retorne a matriz escalonada triangular superior, expandida com $b$, em $A =
\begin{bmatrix} A & b \end{bmatrix}$, do lado esquerdo da function, a partir das
entradas $(n, A)$, onde $n$ é o nº de equações e $A = \begin{bmatrix} A_0 & b
\end{bmatrix}$ é a matriz expandida original de um sistema genérico $A_0 \cdot x
= b$.

\end{smallitem}

\item Dado o sistema linear com 4 tipos de equações:
\begin{align*}
\begin{cases}
x_i - x_{i + 1} = 0.1 &
i = 1 \\
- x_{i - 1} + 2x_i - x_{i + 1} = 0.1 &
i = 2, \dots, n_1 - 1 \\
- x_{i - 2} - x_{i - 1} + 3x_i - x_{i + 1} = 0.2 &
i = n_1, \dots, n_2 - 1 \\
- x_{i - 1} + 2x_i = 0.3 &
i = n_2
\end{cases}
\end{align*}

\begin{smallitem}

\item Considerando $n_1 = 300$ e $n_2 = 400$, se o sistema for resolvido por
métodos iterativos, a sua convergência será garantida? Justifique sua resposta.

\item Se o sistema acima convergir `'lentamente`' para a solução por métodos
iterativos, como a sua convergência pode ser acelerada? Justifique sua resposta.

\item Se o sistema acima convergir `'oscilando`' para a solução por métodos
iterativos, como a sua convergência pode ser acelerada? Justifique sua resposta.

\item Determine a solução $x$ e o resíduo máximo das equações, do sistema acima,
para $n_1 = 3$ e $n_2 = 4$, pelo método de Gauss (sem pivotação);

\item Determine a solução do sistema acima, para $n_1 = 3$ e $n_2 = 4$, com erro
máximo estimado por $max(\vert x(i)- x_i(i) \vert)$ de sua escolha, pelo método
de Gauss--Seidel (sem fator de sub-relaxação).

\end{smallitem}

\item Dado o sistema linear com 4 tipos de equações:
\begin{align*}
\begin{cases}
x_i - x_{i + 1} = 0.1 &
i = 1 \\
- x_{i - 1} + 4x_i - x_{i + 1} = 0.1 &
i = 2, \dots, n_1 - 1 \\
- x_{i - 2} - x_{i - 1} + 2x_i = 0.2 &
i = n_1, \dots, n_2 - 1 \\
- x_{i - 1} + 2x_i = 0.3 &
i = n_2
\end{cases}
\end{align*}

\begin{smallitem}

\item Considerando $n_1 = 300$ e $n_2 = 400$, se o sistema for resolvido por
métodos iterativos, a sua convergência será garantida? Justifique sua resposta.

\item Se o sistema acima convergir de forma `'oscilatória`' ao longo de um
processo iterativo, como a sua convergência pode ser acelerada? Justifique sua
resposta.

\item Determine as matrizes $L$ e $U$ decompostas de $A$, tal que $L \cdot U =
A$, referente ao sistema acima para $n_1 = 3$ e $n_2 = 4$, pelo método de Crout
(sem pivotação);

\item Determine a solução do sistema acima, para $n_1 = 3$ e $n_2 = 4$, com erro
máximo estimado por $max(\vert x(i)- x_i(i) \vert) < 2 \cdot 10^{-2}$, pelo
método de Gauss--Seidel (sem fator de sub-relaxação), a partir da solução
inicial NULA.

\end{smallitem}

\item Dado o seguinte sistema linear:

\begin{smallitem}

\item Se este sistema for resolvido por métodos DIRETOS, verifique se é um
sistema mal-condicionado. Justifique;

\item Que cuidados devemos tomar adicionalmente para resolver sistemas mal-
condicionados?

\item Determine a solução $x$ do sistema acima pelo método de Gauss;

\item Determine o resíduo máximo do sistema acima com a solução $x$ obtida acima
e avalie se este resíduo é satisfatório (de acordo com o número de dígitos
significativos que usar);

\item Monte um algoritmo genérico, tipo \verb!Ab = fescalonamento(n, Ab)!, que
determine e retorne a matriz $Ab$ triangularizada à esquerda, a partir das
entradas $n$ e $Ab$ originais, $Ab = \begin{bmatrix} A & b \end{bmatrix}$
concatenadas em $n$ linhas e $n + 1$ colunas, onde $A \cdot x = b$.

\item Monte um algoritmo genérico \verb!x = fretrosubtituicao(n, Ab)!, que
determine e retorne a solução $x$ a partir das entradas $n$ e $Ab$, onde
$Ab = \begin{bmatrix} A & b \end{bmatrix}$ contém a matriz triangularizada;

\item Determine a solução $x$ do sistema acima pelo método de Crout (sem
pivotação);

\item Determine o resíduo máximo do sistema acima com a solução $x$ obtida e
avalie se este resíduo é satisfatório (de acordo com o número de dígitos
significativos que usar);

\item Monte um algoritmo \verb!x = fsubstituicoesf(n, A)!, que determine e
retorne a solução $x$ do sistema $A_0 \cdot x = b$, a partir das entradas $n$ e
$A$ ($L \cdot U = A_0$ e $A = \begin{bmatrix} L $\textbackslash$ U & b
\end{bmatrix}$, matriz expandida que contém $L$ e $U$ (sobrepostas) e
concatenadas com $b$ em $n$ linhas e $n + 1$ colunas), resolvendo as duas
substituições $L \cdot c = b$ e $U \cdot x = c$;

\item Monte um algoritmo \verb!LU = fLU(n, A)!, que determine e retorne a matriz
expandida decomposta $LU$, com as matrizes $L$ e $U$ (sobrepostas) e
concatenadas com $b$, $LU = \begin{bmatrix} L $\textbackslash$ U & b
\end{bmatrix}$, com pivotação parcial interna, através da função
\verb!A = fpivotacao(n, A, k)! - que determina a matriz expandida $A$ com linha
$k$ com o maior coeficiente em módulo na coluna $k$, onde
$A = \begin{bmatrix} A_0 & b \end{bmatrix}$ e $A_0 \cdot x = b$.

\end{smallitem}

\item Dado o seguinte sistema linear:

\begin{smallitem}

\item Considerando $n_1 = 3000$ e $n_2 = 4000$ com $n_2$ equações, se o sistema
for resolvido por métodos iterativos, a sua convergência será garantida?
Justifique sua resposta.

\item Considerando o sistema acima, é recomendado testar o uso de fatores de sub
ou sobre relaxação na sua resolução por métodos iterativos? Justifique sua
resposta.

\item Monte um algoritmo que determine a solução do sistema acima, para $n_1 =
3000$ e $n_2 = 4000$ equações, com erro máximo estimado por $max(\vert x(i) -
x_a(i) \vert) < 1 \cdot 10^{-6}$, pelo método de Gauss--Seidel com fator de
sub-relaxação 0.5, a partir da solução inicial UNITÁRIA.

\item Monte um algoritmo que determine o erro de truncamento exato da solução
$x$ do sistema acima, para $n_1 = 3000$ e $n_2 = 4000$ equações, com critério de
parada $max(\vert x(i) - x_a(i) \vert) < 1 \cdot 10^{-6}$, pelo método de
Gauss--Seidel com fator de sub-relaxação 0.5, a partir da solução inicial
UNITÁRIA.

\item Determine a solução do sistema acima, para $n_1 = 3$ e $n_2 = 4$ equações,
com erro máximo estimado por $max(\vert x(i) - x_a(i) \vert)$ de sua escolha,
pelo método de Gauss--Seidel (sem fator de sub-relaxação), a partir da solução
inicial UNITÁRIA.

\end{smallitem}

\item Elabore algoritmos que:

\begin{smallitem}

\item Forneça a matriz pivotada $\begin{bmatrix} A & b \end{bmatrix}$ através da
Pivotação Parcial de uma matriz expandida $\begin{bmatrix} A & b \end{bmatrix}$
para uma linha genérica $k$;

\item Forneça a matriz $\begin{bmatrix} A & b \end{bmatrix}$ triangularizada
pelo método de Gauss de uma matriz expandida $\begin{bmatrix} A & b
\end{bmatrix}$ genérica;

\item Forneça a solução $S = \{x_i\}$ de um sistema escrito em forma de matriz
expandida já triangularizada $\begin{bmatrix} A & b \end{bmatrix}$, pelo método
da Retrosubstituição de Gauss;

\item Forneça a matriz $\begin{bmatrix} A & b \end{bmatrix}$, decomposta em $L$
e $U$, pelo método de Crout a partir de uma matriz expandida $\begin{bmatrix} A
& b \end{bmatrix}$;

\item Forneça a solução $S = \{x_i\}$, a partir de um sistema cuja matriz de
coeficientes já está decomposta em $L$ e $U$ pelo método de Crout, onde $L$ e
$U$ estão armazenadas sobrepostas em $A$ na mesma matriz expandida
$\begin{bmatrix} A & b \end{bmatrix}$;

\item Calcule o resíduo máximo das equações de um sistema
$\begin{bmatrix} A_0 & b \end{bmatrix}$ original para uma solução $S = \{x_i\}$.

\end{smallitem}

\item Determine a solução $x$ do seguinte sistema composto por uma matriz
tridiagonal pelo método de Gauss otimizado pelo algoritmo de Thomas:
\begin{align*}
\begin{cases}
x_1 - x_2 = 5 \\
3x_2 - x_3 = 4 \\
x_2 - 2x_3 + x_4 = 10 \\
x_3 - x_4 + x_5 = 3 \\
x_4 + x_5 = -3
\end{cases}
\end{align*}

\item Monte um algoritmo que determine a solução do sistema acima pelo método de
Gauss otimizado pelo algoritmo de Thomas.

\item Dado o sistema linear:
\begin{align*}
\begin{cases}
x_1 - x_2 = 150 \\
x_{i - 1} - 3x_i + x_{i + 1} - x_{i + 10000} = 100 & i = 2, \dots, 2999 \\
x_{i - 1000} - 2.1x_i + x_{i + 1} & i = 3000, \dots, 4999 \\
x_{4999} - x_{5000} = 300
\end{cases}
\end{align*}

\begin{smallitem}

\item Analise o sistema anterior de 5000 equações e escolha um método adequado
para sua resolução. Justifique sua resposta.

\item Se o sistema for resolvido por métodos iterativos, a sua convergência é
garantida?

\item Elabore um algoritmo otimizado para obter sua solução com $max(\vert
x_i^{k + 1} - x_i^k \vert) \leq \epsilon \; (\epsilon = 1 \cdot 10^{-5})$ pelo
método de Gauss--Seidel com sub-relaxação de $\lambda = 0.8$, a partir da
solução inicial $(0, 0, 0, \dots, 0)$.

\end{smallitem}

\item Dado o seguinte sistema linear:
\begin{align*}
\begin{cases}
x_1 + 2x_2 + x_3 = 4 \\
x_1 + 0.1x_2 + x_3 = -3 \\
4x_1 + x_2 + 2x_3 = -1
\end{cases}
\end{align*}

\begin{smallitem}

\item Verifique se este sistema possui diagonal dominante;

\item Efetue uma pivotação parcial em todos as linhas;

\item Verifique novamente se o sistema pivotado apresenta diagonal dominante;

\item Resolva o sistema pivotado pelo método de Gauss-Seidel, com critério
$max(\vert x_i^{k + 1} - x_i^k \vert) \leq \epsilon \; (\epsilon = 1 \cdot
10^{-2})$ partindo da solução inicial $(0, 0, 0)$.

\end{smallitem}

\item Responda:

\begin{smallitem}

\item O que é pivotação parcial e para que serve?

\item A pivotação parcial ajuda a minimizar o acúmulo de erros de
arredonamento ao longo do processo de escalonamento? Justifique.

\item Quais são os requisitos para que um sistema de equações convirja
garantidamente?

\item Quais são os requisitos para que um sistema de equações convirja
rapidamente e sem oscilações?

\end{smallitem}

\item Sabendo que um computador $X$ opera $10^6$  operações em ponto flutuante
por segundo, e que o número total de operações aritméticas envolvidas no
método de Gauss é da ordem de $O(\frac{2}{3} n^3)$ operações:

\begin{smallitem}

\item Quanto tempo de CPU será necessário para resolver um sistema de $n =
1000$ equações neste mesmo computador?

\item Quantos MB (megabytes) são necessários para armazenar um sistema de $n =
1000$ equações utilizando variáveis double (64 bits = 8 B) na forma de matriz
expandida?

\end{smallitem}

\item Dado o seguinte sistema de equações:
\begin{align*}
\begin{cases}
2x_1 - x_2 + x_3 = -1 \\
x_1 - 2x_2 + x_3 = 1 \\
x_1 - 0.1x_2 - x_3 = 3
\end{cases}
\end{align*}

\begin{smallitem}

\item Verifique se o sistema acima é mal condicionado. Justifique;

\item Determine as matrizes $L$ e $U$ e a solução $S = \{x1, x2, x3\}$ do
sistema $A \cdot x = b$ acima pelo método de decomposição $L$ e $U$ de CROUT;

\item Calcule o resíduo máximo da solução $S$ obtida acima.

\end{smallitem}

\item Dado o seguinte sistema linear:
\begin{align*}
\begin{cases}
x_1 + x_3 = 1 \\
x_{i - 1} + 2x_i + x_{i + 1} = 3 & i = 2, \dots, n - 1 \\
x_{n - 1} + 2x_n = 3
\end{cases}
\end{align*}

\begin{smallitem}

\item Considerando $n = 1000$ equações, se o sistema for resolvido por métodos
iterativos, a sua convergência será garantida? Justifique sua resposta.

\item Considerando $n = 4$ equações, determine a solução do sistema, com erro
máximo estimado por $max(\vert x_i - x_i^a \vert) \leq \epsilon$, pelo método
de Gauss--Seidel, com fator de sub-relaxação $\lambda = 0.8$ a partir da
solução inicial unitária ($\epsilon$ de sua escolha). Defina o erro
encontrado.

\item Monte um algoritmo que determine a solução do sistema acima, para $n =
1000$ equações, com critério de parada $max(\vert x_i - x_i^a \vert) \leq
0.000001$, pelo método de Gauss--Seidel, com fator de sub-relaxação $\lambda =
0.8$ a partir da solução inicial unitária. Calcule os erros exatos de cada
$x(i)$ da solução.

\end{smallitem}

\item Dado o seguinte sistema de equações:
\begin{align*}
\begin{cases}
x_1 + x_2 + 1.5x_3 = 4 \\
2x_1 + 0.5x_3 = -3 \\
3x_1 + x_2 + 2x_3 = 1
\end{cases}
\end{align*}

\begin{smallitem}

\item Verifique se o sistema acima é mal-condicionado. Justifique;

\item Determine uma solução do sistema acima pelo método de GAUSS
(triangularização e retrosubstituição), com pivotação parcial;

\item Avalie os resíduos finais das equações e verifique se a solução obtida
tem uma exatidão satisfatória (de acordo com o número de dígitos adotado).

\end{smallitem}

\item Dado o seguinte sistema de equações:
\begin{align*}
\begin{cases}
x_1 + x_2 + 1.5x_3 = 4 \\
2x_1 + 0.01x_2 + 0.5x_3 = -3 \\
3x_1 + x_2 + 2x_3 = 1
\end{cases}
\end{align*}

\begin{smallitem}

\item Verifique se o sistema acima é mal-condicionado. Que cuidados são
necessários para se resolver um sistema mal condicionado com exatidão?
Justifique;

\item Determine a solução do sistema acima pelo método de Crout (decomposição
$LU$);

\item Avalie os resíduos finais das equações e verifique se a solução obtida
tem uma exatidão satisfatória (de acordo com o número de dígitos adotado).

\end{smallitem}

\item Dado o sistema linear para $n = 1000$ equações:
\begin{align*}
\begin{cases}
x_i + x_{i + 1} = 150 & i = 1 \\
x_{i - 1} + 9x_i + x_{i + 1} + x_{i + 100} = 100 &
i = 2, \dots, \frac{n}{2} \\
x_{i - 100} + x_{i - 1} + 9x_i + x_{i + 1} = 200 &
i = \frac{n}{2} + 1, \dots, n - 1 \\
x_{i - 1} + x_i = 300 & i = n
\end{cases}
\end{align*}

\begin{smallitem}

\item Esse sistema pode ser resolvido pelo método de Gauss otimizado, para
matriz banda tridiagonal? Justifique;

\item Monte um algoritmo otimizado, que calcule e imprima uma solução
satisfatória (escolha um erro adequado) para as $n = 1000$ incógnitas, pelo
método que você presuma efetuar o menor número de operações artiméticas em
ponto flutuante. Justifique a escolha do método adotado. Escolha.

\end{smallitem}

\item Dado o sistema linear com 4 tipos de equações:
\begin{align*}
\begin{cases}
x_i - x_{i + 1} = 0.1 & i = 1 \\
- x_{i - 1} + 4x_i - x_{i + 1} = 0.1 & i = 2, \dots, n_1 - 1 \\
- x_{i - 1} + 2x_i - x_{i + 1} = 0.2 & i = n_1, \dots, n_2 - 1 \\
- x_{i - 1} + 2x_i = 0.3 & i = n_2
\end{cases}
\end{align*}

\begin{smallitem}

\item Considerando $n_1 = 300$ e $n_2 = 500$, se o sistema for resolvido por
métodos iterativos, a sua convergência será garantida? Justifique sua
resposta.

\item Se o sistema acima convergir lentamente ao longo de um processo
iterativo, como a sua convergência pode ser acelerada? Justifique sua
resposta.

\item Monte um algoritmo otimizado, que determine e imprima a solução $x$ do
sistema acima pelo método de Gauss otimizado para matrizes tridiagonais.

\item Monte um algoritmo otimizado, que determine e imprima a solução $x$ do
sistema acima pelo método de Gauss--Seidel, sem fator de relaxação, com
critério de parada $max(\vert x(i) - x_i(i) \vert) < 1 \cdot 10^{-4}$.

\item Explique como você calcularia o erro de truncamento existente na solução
$S$, aproximada pelo método de Gauss--Seidel em $d$, utilizando variáveis de
64 bits.

\end{smallitem}

\item Dados os $m = 4$ sistemas $A \cdot x = b^m$ cada um de $n = 3$ equações
abaixo, com a mesma matriz $A$ e com $m = 4$ termos independentes $b_i^m$
diferentes:
\begin{align*}
m = 1 \longrightarrow &
\begin{cases}
4_x1 + x_2 + 2x_3 = 1 \\
x_1 - 2x_2 + x_3 = 4 \\
x_1 + 0.1x_2 - x_3 = -3
\end{cases} &
m = 2 \longrightarrow &
\begin{cases}
4_x1 + x_2 + 2x_3 = -1 \\
x_1 - 2x_2 + x_3 = 10 \\
x_1 + 0.1x_2 - x_3 = -3
\end{cases} \\
m = 3 \longrightarrow &
\begin{cases}
4_x1 + x_2 + 2x_3 = 7 \\
x_1 - 2x_2 + x_3 = -4 \\
x_1 + 0.1x_2 - x_3 = 3
\end{cases} &
m = 4 \longrightarrow &
\begin{cases}
4_x1 + x_2 + 2x_3 = 5 \\
x_1 - 2x_2 + x_3 = 3 \\
x_1 + 0.1x_2 - x_3 = 1
\end{cases} \\
A = &
\begin{bmatrix}
4 & 1 & 2 \\
1 & -2 & 1 \\
1 & 0.1 & -1
\end{bmatrix} &
b = &
\begin{bmatrix}
1 & -1 & 7 & 5 \\
4 & 10 & -4 & 3 \\
-3 & -3 & 3 & 1
\end{bmatrix}
\end{align*}

Monte um algoritmo genérico, que determine as $m$ soluções $S_m = \{x_1^m,
x_2^m, x_3^m\}$ dos $m$ sistemas $A \cdot x = b^m$ acima, através das duas
substituições $L \cdot c^m = b^m$ e $U \cdot x^m = c^m$ propostas por Crout.

Está disponível a função \verb![L U] = fLUCrout(n, A)!, que calcula e retorna
as matrizes $L_{n x n}$ e $U_{n x n}$ (sem pivotação) sobrepostas na mesma
matriz $LU$, por decomposição da matriz $A_{n x n}$ pelo método de CROUT ($A =
L\ cdot U$).

\item  Dado o sistema linear com 4 tipos de equações:
\begin{align*}
\begin{cases}
x_i - x_{i + 1} = 0.1 & i = 1 \\
- x_{i - 1} + 4x_i - x_{i + 1} = 0.1 & i = 2, \dots, n_1 - 1 \\
- x_{i - 1} + 2x_i - x_{i + 1} = 0.2 & i = n_1, \dots, n_2 - 1 \\
- x_{i - 1} + 2x_i = 0.3 & i = n_2
\end{cases}
\end{align*}
onde $n_1 = 300$ e $n_2 = 500$.

\begin{smallitem}

\item Se o sistema for resolvido por métodos iterativos, a sua convergência
será garantida? Justifique sua resposta.

\item Se o sistema acima convergir `'oscilando ou lentamente` ao longo de um
processo iterativo, como a sua convergência pode ser acelerada? Justifique sua
resposta.

\item Monte um algoritmo otimizado, que determine e imprima a solução $x$ do
sistema acima pelo método de Gauss--Seidel, com fator de relaxação $f = 1.4$,
com critério de parada $max(\vert x(i) - x_i(i) \vert) < 1 \cdot 10^{-4}$ e
que determine e imprima o erro de truncamento máximo da solução $x$ obtida
(utilize variáveis de 64 bits para minimizar os efeitos dos arredondamentos).

\end{smallitem}

\end{enumerate}

\end{document}
