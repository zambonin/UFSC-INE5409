\documentclass[12pt]{article}

\usepackage[brazil]{babel}
\usepackage[T1]{fontenc}
\usepackage[a4paper, margin=1.5cm]{geometry}
\usepackage[colorlinks, urlcolor=blue]{hyperref}
\usepackage[utf8]{inputenc}
\usepackage{amsmath, enumitem}

\newenvironment{smallitem}{
    \vspace{-2mm}
    \begin{enumerate}
    \setlength{\parskip}{0pt}
    \setlength{\itemsep}{2pt}
}{
    \vspace{-2mm}
    \end{enumerate}
}

\begin{document}

{\bf \noindent Universidade Federal de Santa Catarina - UFSC \\
Centro Tecnológico - CTC \\
Departamento de Informática e Estatística - INE \\

\noindent Lista de Exercícios 1 - INE5202/INE5232/INE5409 \\
Prof. Sérgio Peters (\texttt{sergio.peters@ufsc.br})}

\begin{enumerate}[label=\textbf{\arabic*})]

\item Efetue as seguintes conversões de base:

\begin{smallitem}

\item $(10.1011)_2 = (?)_{10}$

\item $(10.57)_{10} = (?)_2$

\end{smallitem}

\item Implemente o algoritmo abaixo em um computador com processamento
numérico.
\begin{verbatim}
    Início
        inteiro n
        real x
        leia n
        x = 1/n
        imprimir ‘valor inicial x: ’, x (com 30 significativos)
        Para i = 1 até 1
            x = (n + 1) x - 1
            imprimir i , x (com 30 significativos)
        Fim para
    Fim\end{verbatim}
Obs.: Teste $n = 2, 3, 10, 16$ e avalie a evolução de $x$ com o número de
repetições $i$.  Note que se $x = \frac{1}{n}$ tiver representação exata,
$x = (n + 1) \cdot \frac{1}{n} - 1 = 1 + \frac{1}{n} - 1 = \frac{1}{n} = x$.

Logo, o valor de $x$ não deveria se alterar com estes cálculos sucessivos, se a
primeira definição de $x$ for exata. Explique a razão das diferenças nos valores
de $x$, entre um cálculo e outro, para alguns $n$.

\item (opcional) Existe uma base onde todo número racional tem representação
finita, de acordo com
\href{https://en.wikipedia.org/wiki/Georg_Cantor}{Georg Cantor}, “todo racional
tem representação finita na base fatorial”.

Conceitualmente, a base fatorial é semelhante à decimal com a diferença de que
em um número $X_{F!} = (a_n a_{n-1} \dots a_1.a_{-1} a_{-2} \dots a_{-m})_{F!}$
cada $a_i$ só pode assumir um valor do intervalo $0 < a_i < \lvert i \rvert$,
onde a parte inteira é representada por
\begin{equation*}
(a_n a_{n-1} \dots a_1)_{F!} = \left( \sum_{i=n}^1 a_i i! \right)_{10}
\end{equation*}
e a parte fracionária, por
\begin{equation*}
(0.a_{-1} a_{-2} \dots a_{-m})_{F!} = \left( \sum_{i=1}^m
\frac{a_{-i}}{(i+1)!} \right)_{10}.
\end{equation*}
Observe que $(4321)_{F!} = (119)_{10}$ terá como seu sucessor $(10000)_{F!} =
(120)_{10}$.

Represente os números na forma fatorada e reconverta para a base decimal:

\begin{smallitem}

\item $(3021)_{F!} = 3 \cdot 4! + 0 \cdot 3! + 2 \cdot 2! + 1 \cdot 1! =
(77)_{10}$

\item $(4321)_{F!} = (?)_{10}$

\item $(10000)_{F!} = (?)_{10}$

\item $(0.02)_{F!} = \frac{0}{2!} + \frac{2}{3!} = (\frac{2}{6})_{10} =
(\frac{1}{3})_{10} = (0.333333 \dots)_{10}$

\item $(0.113)_{F!} = (?)_{10}$

\item $(321.123)_{F!} = (?)_{10}$

\end{smallitem}

Note que nos exercícios (d), (e) e (f) tem-se representações exatas de números
racionais, que na base decimal são dízimas periódicas.

\item Converta os números a seguir para as bases, na ordem indicada:

\begin{smallitem}

\item $(10111.1101)_2 = (?)_{16} = (?)_{10}$

\item $(BD.0E)_{16} = (?)_2 = (?)_{10}$

\item $(41.1)_{10} = (?)_2 = (?)_{16}$

\end{smallitem}

Obs.: Indique onde poderá haver perda de dígitos significativos, considerando um
número limitado de dígitos representáveis.

\item Na representação $F(2, 3, -3, +3)$ com $d_1 \neq 0$ antes da vírgula, não
polarizada, calcule:

\begin{smallitem}

\item O número de mantissas representáveis;

\item O número de expoentes representáveis;

\item O número de elementos representáveis;

\item Defina as regiões de \textit{underflow} e \textit{overflow};

\item Estime a precisão decimal equivalente;

\item Proponha uma transformação da representação $F$ apresentada em uma nova
com polarização que utilize os limites dos 3 bits totais reservados ao expoente.

\end{smallitem}

\item Na representação $F(2,3,0,7)$ com $d_1 \neq 0$ antes da vírgula e
polarização $p = +3$ calcule:

\begin{smallitem}

\item O número de mantissas representáveis;

\item O número de expoentes representáveis;

\item O número de elementos representáveis;

\item Defina as regiões de \textit{underflow} e \textit{overflow};

\item Estime a precisão decimal equivalente;

\end{smallitem}

\item Avalie as regiões de \textit{underflow} e \textit{overflow} para a
variável de 64 bits, e teste os seus limites em algum compilador numérico (C,
Java, \dots) ou em um software matemático livre (se usar Octave, pode usar os
comandos \verb!single()! e \verb!double()!, para simular variáveis).

\item Avalie a precisão decimal equivalente da variável de 64 bits, através das
três formas apresentadas.

\item Simule o algoritmo abaixo:
\begin{verbatim}
    tipo real: e, f, g, h, x, y

    h:= 1/2;
    x:= 2/3 - h;
    y:= 3/5 - h;
    e:= (x + x + x) - h;
    f:= (y + y + y + y + y) - h;
    g:= e/f;
    Imprima h,x,y,e,f,g (com 25 dígitos)\end{verbatim}
Obs: Use variáveis reais de 32 e 64 bits, e explique a causa dos resultados
obtidos para $g$, que deveria ser uma indeterminação $\frac{0}{0}$.

\item Para achar as duas raízes de $x^2 \, + \, 62.10x \, + \, 1 = 0$,
utilizando o operador aritmético $F(10, 4, -99, +99)$ (quatro dígitos
significativos nas operações):

\begin{smallitem}

\item Use a fórmula de Bhaskara normal;

\item Use a fórmula de Bhaskara normal e racionalizada, sempre com adição das
parcelas;

\item Avalie os erros relativos nas duas formas de avaliação das raízes, sabendo
que os seus valores exatos são $x_1 = -0.0161072$ e $x_2 = - 62.0839$.

\end{smallitem}

\item Dadas algumas estimativas do valor exato e de valores aproximados
numericamente em um algoritmo, avalie o erro absoluto, relativo e percentual,
existente nas seguintes situações:

\begin {smallitem}

\item Valor aproximado $= 1102.345$ e Valor exato $= 1100.9$.

\item Valor aproximado $= 0.01245$ e Valor exato $= 0.0119$.

\item Verifique que o erro absoluto obtido, segundo as várias formas de
avaliação, pode não refletir a realidade.

\end{smallitem}

\item Dado o número decimal $x = -(10.05)_{10}$:

\begin{smallitem}

\item Calcule os 32 bits da variável IEEE 754 que armazena $x$;

\item Calcule o erro de arredondamento percentual exato gerado no armazenamento
de $x$ em binário.

\end{smallitem}

\item Dado o número binário $x = (10000000010000000000000000000001)_2$,
armazenado no padrão IEEE de 32 bits (4B), calcule o decimal $x$ correspondente.

\item Dado o número decimal $y = -(1.51 \cdot 10^{+37})_{10}$:

\begin{smallitem}

\item Calcule os 32 bits da variável $y$ armazenada no padrão IEEE de 32 bits
(4B). Mostre explicitamente a parcela binária que foi arredondada;

\item Calcule o erro exato relativo percentual de arredondamento gerado na
conversão de $y$ decimal para binário da variável IEEE de 32 bits.

\end{smallitem}

\item Dado o número decimal exato $y = -(1.1 \cdot 10^{-41})_{10}$:

\begin{smallitem}

\item Calcule os 32 bits da variável $y$ armazenada no padrão IEEE. Mostre
explicitamente a parcela binária que foi arredondada;

\item Calcule o erro exato relativo percentual de arredondamento gerado na
conversão de $y$ decimal para binário da variável IEEE de 32 bits.

\end{smallitem}

\item Explique como você calcularia em computador o erro relativo aos
arredondamentos acumulados, nos armazenamentos e nas operações usadas, para se
obter uma solução numérica qualquer utilizando variáveis de 32 bits. Está
disponível também a variável de 64 bits.

\item Implemente um algoritmo que calcule e imprima, o erro de arredondamento da
função $exp(x)$ calculada na variável 32 bits, através de aproximação por série
de Maclaurin com $n = 4$ em $x = -0.111$, conforme expressão abaixo:
\begin{equation*}
exp(x) = e^x \cong 1 + \frac{x}{1!} + \frac{x^2}{2!} + \frac{x^3}{3!} +
\frac{x^4}{4!} + \dots + \frac{x^n}{n!}
\end{equation*}

\item Implemente um algoritmo que calcule e imprima, o erro de arredondamento da
função $exp(x)$ calculada na variável \verb!double!, através de aproximação por
série de Maclaurin com $n = 4$ em $x = -0.111$, conforme expressão abaixo:
\begin{equation*}
exp(x) = e^x \cong 1 + \frac{x}{1!} + \frac{x^2}{2!} + \frac{x^3}{3!} +
\frac{x^4}{4!} + \dots + \frac{x^n}{n!}
\end{equation*}

\end{enumerate}

\end{document}
